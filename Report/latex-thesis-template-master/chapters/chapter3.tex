\chapter{Results}

\section{Coin-cell cycling (Task 3.1)}

\subsection{Potential\textendash capacity profiles}
\autoref{fig:T31_potential_vs_capacity} overlays the specific capacity\,\textit{vs.}\,potential curves at the three C\textendash rates. At the low rate (0.1\,C), the discharge capacity is highest and the (de)lithiation plateaus are most clearly developed. With increasing rate (1\,C and 2\,C), polarization grows (larger charge\textendash discharge hysteresis) and the accessible capacity decreases, consistent with kinetic and transport limitations. The chosen voltage window (2.4\,V to 4.2\,V) spans the practical operating region of the NCM\,811\,\textbar\,graphite full cell; at low rate it enables utilization of most of the available capacity, while at higher rates kinetic limitations dominate the accessible capacity.

\begin{figure}[h]
    \centering
    \includesvg[width=0.8\textwidth]{images/T3.1_Potential_vs_Capacity.svg}
    \caption{Potential\,\textit{vs.}\,specific capacity at 0.1\,C, 1\,C and 2\,C for the assembled coin cell. Higher C\textendash rates exhibit increased polarization and reduced accessible capacity compared to 0.1\,C.}
    \label{fig:T31_potential_vs_capacity}
    \end{figure}

The experimental discharge capacities at 0.1\,C reach approximately 227\,mAh/g, which compares favorably to the theoretical capacity of graphite (372\,mAh/g for LiC$_6$ formation). The capacity shows a clear dependence on C\textendash rate, with decreasing accessible capacity at higher rates due to kinetic limitations and increased polarization.

\subsection{Incremental capacity (dQ/dE) analysis}
The cathodic incremental capacity analysis reveals distinct electrochemical processes at each C\textendash rate.

\begin{figure}[h]
\centering
\begin{subfigure}[b]{0.48\textwidth}
\centering
\includesvg[width=\textwidth]{images/T3.1_dQdE_0.1C.svg}
\caption{0.1\,C}
\label{fig:T31_dQdE_01C}
\end{subfigure}
\hfill
\begin{subfigure}[b]{0.48\textwidth}
\centering
\includesvg[width=\textwidth]{images/T3.1_dQdE_1C.svg}
\caption{1\,C}
\label{fig:T31_dQdE_1C}
\end{subfigure}
\\[0.5em]
\begin{subfigure}[b]{0.48\textwidth}
\centering
\includesvg[width=\textwidth]{images/T3.1_dQdE_2C.svg}
\caption{2\,C}
\label{fig:T31_dQdE_2C}
\end{subfigure}
\caption{Incremental capacity (dQ/dE)\,\textit{vs.}\,potential at different C\textendash rates. At 0.1\,C, sharp, well\textendash resolved peaks reveal distinct graphite staging transitions and NMC811 cathode redox processes. With increasing rate, peaks broaden and decrease in amplitude, with high\textendash voltage features showing progressive suppression.}
\label{fig:T31_dQdE_all}
\end{figure}

\autoref{fig:T31_dQdE_01C} shows the 0.1\,C discharge profile, where several sharp, well\textendash resolved peaks appear between 3.0\textendash 4.1\,V. A dominant group around 3.4\textendash 3.8\,V corresponds to staged Li$^+$ intercalation in graphite (progression toward stage~1, LiC$_6$), while a shoulder near 3.9\textendash 4.1\,V reflects the high\textendash voltage redox of the NMC811 cathode. These features align with the plateaus observed in \autoref{fig:T31_potential_vs_capacity}.

At 1\,C (\autoref{fig:T31_dQdE_1C}), the peaks begin to broaden and decrease in amplitude, with the highest\textendash voltage features showing the most significant suppression. The graphite staging peaks remain discernible but are less sharp, indicating increased kinetic limitations while maintaining reasonable reversibility.

The 2\,C profile (\autoref{fig:T31_dQdE_2C}) demonstrates substantial peak broadening and amplitude reduction. The high\textendash voltage cathode features are largely suppressed, and the graphite staging peaks merge into broader features, reflecting transport\textendash limited behavior at this rate.

From these individual rate analyses we conclude that rate\textendash induced polarization progressively dominates with increasing C\textendash rate, merging graphite staging peaks and suppressing high\textendash voltage features. No additional peaks emerge at high rate within 2.4\textendash 4.2\,V, consistent with the high Coulombic efficiency and absence of parasitic processes. Graphite intercalation steps remain discernible up to 1\,C, indicating acceptable anode kinetics; at 2\,C they become transport\textendash limited.

\subsection{Coulombic efficiency}
The Coulombic efficiency (CE) was evaluated as the ratio of discharge to charge capacity for each cycle using Equation~\ref{eq:ce}:

\begin{equation}
CE = \frac{Q_{discharge}}{Q_{charge}} \times 100\%
\label{eq:ce}
\end{equation}

The CE values from the cycling data show distinct behavior at different C\textendash rates. \autoref{tab:ce_values} presents the complete CE data for all cycles at each C\textendash rate.

\begin{table}[h]
\centering
\caption{Coulombic efficiency values for all cycles at different C\textendash rates}
\label{tab:ce_values}
\begin{subtable}[h]{0.32\textwidth}
\centering
\caption{0.1\,C}
\begin{tabular}{cc}
\toprule
Cycle & CE (\%) \\
\midrule
1 & 39.1 \\
2 & 101.8 \\
3 & 98.8 \\
4 & 100.1 \\
5 & 100.0 \\
6 & 99.9 \\
7 & 99.7 \\
8 & 99.7 \\
9 & 99.7 \\
10 & 99.7 \\
\bottomrule
\end{tabular}
\end{subtable}
\hfill
\begin{subtable}[h]{0.32\textwidth}
\centering
\caption{1\,C}
\begin{tabular}{cc}
\toprule
Cycle & CE (\%) \\
\midrule
11 & 95.7 \\
12 & 98.9 \\
13 & 99.1 \\
14 & 99.1 \\
15 & 99.1 \\
16 & 99.2 \\
17 & 99.2 \\
18 & 99.2 \\
19 & 99.3 \\
20 & 99.3 \\
\bottomrule
\end{tabular}
\end{subtable}
\hfill
\begin{subtable}[h]{0.32\textwidth}
\centering
\caption{2\,C}
\begin{tabular}{cc}
\toprule
Cycle & CE (\%) \\
\midrule
21 & 98.7 \\
22 & 99.0 \\
23 & 98.9 \\
24 & 98.8 \\
25 & 98.8 \\
26 & 98.7 \\
27 & 98.6 \\
28 & 98.6 \\
29 & 98.6 \\
30 & 98.6 \\
\bottomrule
\end{tabular}
\end{subtable}
\end{table}

At 0.1\,C, the initial cycle shows a low CE of 39.1\% due to SEI formation and initial activation processes. However, subsequent cycles rapidly stabilize with CE values approaching 100\% (cycles 2\textendash 10: 98.8\textendash 100.1\%), indicating excellent reversibility once the SEI is established.

At 1\,C, the CE remains consistently high throughout cycling (95.7\textendash 99.3\%), with slight variations attributed to increased polarization effects. The 2\,C rate shows similar high CE values (98.6\textendash 99.0\%), demonstrating that the cell maintains good reversibility even at elevated current densities.

The CE shows minimal dependence on C\textendash rate once the initial formation cycles are completed, indicating that the chosen voltage window (2.4\textendash 4.2\,V) allows for stable operation across all tested rates without significant side reactions or electrolyte decomposition.
