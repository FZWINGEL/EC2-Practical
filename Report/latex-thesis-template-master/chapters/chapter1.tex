\chapter{Introduction}
\setcounter{page}{1}


Lithium ion batteries (LIBs) have become the backbone of modern portable electronics and electric vehicles.\supercite{Goodenough2013} When Sony commercialized the first LIB in 1991,\supercite{VanNoorden2014} it marked the beginning of a transformation in energy storage that would eventually earn Whittingham, Goodenough, and Yoshino the 2019 Nobel Prize in Chemistry.\supercite{Goodenough2013} The advantages of LIBs over earlier rechargeable systems (higher energy and power densities, higher operating voltages, and longer cycle life)\supercite{VanNoorden2014} have made them the dominant technology in today's applications.

The basic operating principle of a LIB involves two intercalation electrodes (anode and cathode) separated by a lithium-ion conducting electrolyte and porous separator. During discharge, Li$^+$ ions leave the graphite anode and travel through the electrolyte to the cathode (typically a lithium metal oxide), while electrons flow through the external circuit; charging reverses this process. Graphite has emerged as the standard anode material because it accommodates Li$^+$ ions between its graphene layers (forming LiC$_6$) with relatively little structural change. However, the low potential of lithiated graphite lies outside the electrochemical stability window of most organic electrolytes, which leads to electrolyte decomposition and formation of a solid electrolyte interphase (SEI) on the anode surface. This nanoscale passivation layer blocks electron transport while allowing Li$^+$ ions to pass through, though its initial formation irreversibly consumes some lithium. Despite this cost, the SEI prevents ongoing electrolyte breakdown and is critical for long-term battery stability.\supercite{Goodenough2013}

Battery performance, internal resistance, and lifetime all depend heavily on these interfacial processes (SEI formation, charge-transfer kinetics, and ionic transport through the electrolyte). Increasing interfacial resistance or persistent side reactions (such as electrolyte decomposition or electrode degradation) gradually reduce battery capacity by consuming active lithium and raising internal resistance.\supercite{Goodenough2013}

A range of electrochemical characterization techniques exist to probe these interfacial processes. Cyclic voltammetry (CV) measures current as a function of swept voltage, revealing redox reactions and their reversibility. A reversible redox process produces symmetric oxidation and reduction peaks whose separation and shape reflect Nernstian equilibrium and diffusion effects, while kinetic limitations cause peak shifts and distortion.\supercite{Elgrishi2018}

Galvanostatic charge--discharge testing (cycling at constant current) remains the standard technique for evaluating battery capacity, energy efficiency, and rate capability. These measurements yield specific capacity (mAh g$^{-1}$) and Coulombic efficiency (the discharge-to-charge capacity ratio) for each cycle.

Electrochemical impedance spectroscopy (EIS) takes a different approach by applying small AC perturbations across a range of frequencies. This technique excels at separating different interfacial processes by fitting the measured impedance to equivalent circuit models. The high-frequency intercept in a Nyquist plot typically corresponds to electrolyte resistance, while mid- to low-frequency features (semicircles or Warburg slopes) reflect double-layer capacitance, charge-transfer resistance, and diffusion impedance. Fitting these spectra extracts quantitative parameters like charge-transfer resistance, exchange current, and diffusion coefficients. Unlike large-perturbation methods, EIS operates near equilibrium, which allows precise characterization of both fast and slow processes.\supercite{Lazanas2023}

Together, voltammetric, galvanostatic, and impedance techniques provide complementary insights into electrochemical interfaces, identifying thermodynamic, kinetic, and transport properties.


 

